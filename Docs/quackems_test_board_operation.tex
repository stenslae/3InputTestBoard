\documentclass[12pt, letterpaper]{article}
\usepackage{graphicx, amsmath, hyperref, pdfpages, float, helvet}
\renewcommand{\familydefault}{\sfdefault}
\graphicspath{{images/}}
\title{\textbf{QUACKEMS 3-Input Test Board}}
\author{Emma Stensland}
\date{June 19, 2025}
\begin{document}
\begin{figure}
    \centering
    \includegraphics[width=0.5\textwidth]{quackems_logo.png} 
    \label{fig:quackems}
\end{figure}
\vspace{-12pt}
\maketitle
\newpage
\tableofcontents
\newpage
\section{General Description}
\subsection{Introduction}
\hspace{\parindent}The 3-Input Test board analyzes the output of the W1 Silicon Strip 
Detector for \textbf{QUACKEMS}\footnote{Quantitative Charge Kinetic Energy Mass Sensor}. 
The W1 Detector has 16 channels correlating to each silicon strip, 
and 3 of the channels may be tested at a time.\newline 

\textit{\textbf{As of June 
19, 2025, the 3 inputs on the board are soldered to channels 7, 8, and 9.}} 

\newpage
\subsection{Theory of Operation}
\hspace{\parindent}\textbf{W1 Detector:} The W1 silicon strip detector has p-type 
strips and n-type bulk, with metallization on the top strips and bottom surface. 
This means that the detector behaves like a diode, and when the detector is reverse 
biased a depletion zone is made. When a charged particle crosses the depletion zone, 
it ionizes atoms along its track. In the depletion zone, there is no free sharge to accept
the free electrons and holes. Therefore, they drift to the electrodes on opposing surfaces.
The free holes get collected by the strip side of the detector, and generate a charge pulse. 
\newline
\begin{figure}[H]
    \centering
    \includegraphics[width=0.70\textwidth]{detector_diagram_1.png} 
    \caption{Basic Diagram of Silicon Strip Detector}
    \label{fig:ssd}
\end{figure}

\hspace{\parindent}\textbf{Board:} The purpose of the board is to take the input charge 
pulse of the W1 detector and convert it to a voltage pulse signal. 
This is done with a preamplifier and pulse shaper stage.
The charge sensitive preamplifier stage integrates the charge pulse and produces a 
voltage pulse that decays exponentially. The gaussian pulse shaper stage takes the 
tail pulse from the preamplifier and produces a bell curve as the output.
All 16 channels of the detector are AC-coupled. A low-pass filter maintains a steady 
bias voltage, the bias supplies current to the DC detector, and a capacitor between 
the preamplifier input and detector output blocks any DC current.
The test inputs simulate a charge pulse when a square wave is inputted.
The board itself is encased in a grounded metal enclosure that functions as a Faraday cage.
\begin{figure}[H]
    \centering
    \includegraphics[width=1\textwidth]{detector_connections.jpg} 
    \caption{Basic Diagram of Front End System}
    \label{fig:connections}
\end{figure}

\newpage
\subsection{Specifications}
The following specifications are required for proper usage of the board:
\begin{itemize}
    \item Amplifier Supply Voltage:  6\,V--13\,V
    \item Bias Supply: Dependent on W1 Detector, needs to be negative for detector
    to enter depletion.
    \item Grounding: Ensure the metal enclosure is properly grounded 
    during operation to maintain signal integrity and reduce noise.
\end{itemize}

\newpage
\subsection{Board Overview}
\begin{figure}[H]
    \centering
    \includegraphics[width=1\textwidth]{board_overview.jpg} 
    \caption{Top view of PCB}
    \label{fig:pcbimage}
\end{figure}
Each numbered box correlates to the following part of the board:
\begin{enumerate}
    \item W1 Detector Pin Header
    \item Input Test Signal SMA Connectors
    \item W1 Detector Input Signal Jumper Connections
    \item Screw Terminal for Power
    \item Output Signals SMA Connectors
    \item Preamplified Test Point SMA Connectors
\end{enumerate}

\newpage
\section{Connections}
\subsection{W1 Detector}
\hspace{\parindent}This gets attached at the pin header 1 seen in Figure \ref{fig:pcbimage}. The strip side 
of the detector is oriented up with the top of the board, while the ohmic side faces down.
\begin{figure}[H]
    \centering
    \includegraphics[width=0.70\textwidth]{detector_to_board_connection.jpg} 
    \caption{Detector connected to board}
    \label{fig:detectorconnect}
\end{figure}
\subsection{Input Test Signals}
\hspace{\parindent}The SMA connectors supply test signals on the exterior of the enclosure as seen in Figure 
\ref{fig:enclosure1}.
\subsection{Output Signals}
\hspace{\parindent}The SMA connectors read out the output signals on the exterior of the enclosure as seen in 
Figure \ref{fig:enclosure1}.
\begin{figure}[H]
    \centering
    \includegraphics[width=0.75\textwidth]{box_side_1.jpg} 
    \caption{Input and Output Side View of enclosure}
    \label{fig:enclosure1}
\end{figure}

\subsection{Detector Input Signals}
\hspace{\parindent}These signals are soldered onto the detectors channels as described below:
\begin{table}[H]
\centering
\begin{tabular}{|c|c|}
\hline
\textbf{Test Input} & \textbf{Detector Channel} \\
\hline
Input 1 & Channel 7 \\
Input 2 & Channel 8 \\
Input 3 & Channel 9 \\
\hline
\end{tabular}
\caption{Test Input to Detector Channel Mapping}
\end{table}

\subsection{Amp Power}
\hspace{\parindent}To power the amplifiers, a positive power supply in the range of +6\,V to +13\,V should be 
connected to the +6\,V terminal, and a negative supply from -6\,V to -13\,V to the -6\,V 
terminal. For testing, 9\,V batteries are to be connected as pictured in 
Figure \ref{fig:pcbimage2}.
\begin{figure}[H]
    \centering
    \includegraphics[width=0.75\textwidth]{board_batteries.jpg} 
    \caption{Batteries Connected to Board}
    \label{fig:pcbimage2}
\end{figure}

\subsection{Preamplified Test Points}
\hspace{\parindent}The SMA connectors read test points after the signal has been preamplified at the location
pictured in Figure \ref{fig:enclosure2}.
\begin{figure}[H]
    \centering
    \includegraphics[width=0.60\textwidth]{box_side_2.jpg} 
    \caption{Test Point Side View of enclosure}
    \label{fig:enclosure2}
\end{figure}

\subsection{Bias and Ground}
\hspace{\parindent}The SMA connector on the top of the enclosure has a red and black wire. The black wire connects 
to ground and the red wire connects to bias, as shown in Figure \ref{fig:enclosure3}.
\begin{figure}[H]
    \centering
    \includegraphics[width=0.70\textwidth]{box_side_3.jpg} 
    \caption{Bias Top View of enclosure}
    \label{fig:enclosure3}
\end{figure}

\newpage
\section{Testing Procedure}
In order to verify successful operation of the board, the following procedure should be 
used for channels 1-3.
\begin{enumerate}
    \item Connect the batteries to the board, as pictured in Figure \ref{fig:pcbimage2}.
    \item Connect an SMA connector from the output of a function generator to the input of the 
    channel desired to be tested. The input channels are the bottom row of SMA connections on
    the side of the enclosure, as pictured in Figure \ref{fig:enclosure1}.
    \item Set the function generator to 1\,kHz, with 0\,V offset, and 500\,mV$_{pp}$, 
    and turn the function generator on. This generates a impulse-like excitation after 
    AC coupling, suitable for shaping circuit analysis.
    \item Connect an SMA connector from an oscilloscope to the test point of the channel being
    tested. The test point connections can be found on the side of the enclosure, as seen in 
    Figure \ref{fig:enclosure2}. Adjust the scaling of the oscilloscope as necessary, and the following waveform
    should appear:
    \begin{figure}[H]
        \centering
        \includegraphics[width=0.70\textwidth]{test_pulse.jpg} 
        \caption{Preamplified Input as Seen on Oscilloscope}
        \label{fig:image1}
    \end{figure}
    \item Connect an SMA connector from an oscilloscope to the output of the channel being
    tested. The test point connections can be found on the side of the enclosure, as seen in
    Figure \ref{fig:enclosure1}. At the same scaling as before the output will appear like 
    Figure \ref{fig:image2}, and when zoomed in, the oscilloscope should display Figure \ref{fig:image3}
    \begin{figure}[H]
        \centering
        \includegraphics[width=0.70\textwidth]{output_scope.jpg} 
        \caption{Output as Seen on Zoomed out Image of Oscilloscope}
        \label{fig:image2}
    \end{figure}
    \begin{figure}[H]
        \centering
        \includegraphics[width=0.70\textwidth]{output_zoomed.jpg} 
        \caption{Output as Seen on Zoomed in Image of Oscilloscope}
        \label{fig:image3}
    \end{figure} 
    \item If all waveforms look as expected, the tested channel is working properly. 
    If this test failed, verify all SMA connectors and wires are properly connected, 
    and power is being properly supplied to the board.
\end{enumerate}

\newpage
\section{Design Overview}
\subsection{PCB Layout}
The PCB is a four layer board with each layer serving a purpose as listed:
\begin{enumerate}
    \item Ground Plane, Bias Signals, Surface Mount Signals (Fig. \ref{fig:boardlayer1})
    \item Electrically Shielded I/O Signals of Preamp (Fig. \ref{fig:boardlayer2})
    \item Ground Plane (Fig. \ref{fig:boardlayer3})
    \item Power Lines for Preamp and Shaper (Fig. \ref{fig:boardlayer4})
\end{enumerate}
\newpage
\begin{figure}[H]
    \centering
    \includegraphics[width=1\textwidth]{board_layer1.png} 
    \caption{Layer 1 of 3-Input Test Board}
    \label{fig:boardlayer1}
\end{figure}
\begin{figure}[H]
    \centering
    \includegraphics[width=1\textwidth]{board_layer2.png} 
    \caption{Layer 2 of 3-Input Test Board}
    \label{fig:boardlayer2}
\end{figure}
\begin{figure}[H]
    \centering
    \includegraphics[width=1\textwidth]{board_layer3.png} 
    \caption{Layer 3 of 3-Input Test Board}
    \label{fig:boardlayer3}
\end{figure}
\vspace{-1em}
\begin{figure}[H]
    \centering
    \includegraphics[width=1\textwidth]{board_layer4.png} 
    \caption{Layer 4 of 3-Input Test Board}
    \label{fig:boardlayer4}
\end{figure}

\clearpage
\newpage
\section{References}
The following resources were used:
\begin{itemize}
    \item \hyperlink{https://www.cremat.com/}{Cremat Inc Website (cremat.com)}
    \item \hyperlink{https://indico.cern.ch/event/124392/contributions/1339904/attachments/74582/106976/IntroSilicon.pdf}{CERN Silicon Strip Detector Slide Show}
    \item CR-110-R2.2 Datasheet
    \item CR-200-R2.1 Datasheet
    \item W1 300um Documentation
\end{itemize}

\includepdf[pages={1}]{pdf/CR-110-R2.2.pdf}
\includepdf[pages={1}]{pdf/CR-200-R2.1.pdf}
\includepdf[pages={1,2}]{pdf/W1 Data sheet.pdf}
\newpage
\end{document}